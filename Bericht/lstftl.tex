% Listing style definition for the Lean Theorem Prover.
% Defined by Jeremy Avigad, 2015, by modifying Assia Mahboubi's SSR style.
% Unicode replacements taken from Olivier Verdier's unixode.sty

\lstdefinelanguage{ftl} {

% Anything betweeen $ becomes LaTeX math mode
mathescape=true,
% Comments may or not include Latex commands
texcl=false,

% keywords, list taken from lean-syntax.el
morekeywords=[1]{
Proof, end, Qed
},

% Sorts
morekeywords=[2]{Lemma, Theorem, Axiom, Definition, Signature},

morekeywords=[3]{let, Let, us, such, show, that, If, then, we, have, denote, assume, Assume}


% Spaces are not displayed as a special character
showstringspaces=false,

% keep spaces
keepspaces=true,

% String delimiters

% Size of tabulations
tabsize=1,

% Enables ASCII chars 128 to 255
extendedchars=false,

% Case sensitivity
sensitive=true,

% Automatic breaking of long lines
breaklines=true,

% Default style fors listingsred
basicstyle=\ttfamily,

% Position of captions is bottom
captionpos=b,

% Full flexible columns
columns=[l]fullflexible,


% Style for (listings') identifiers
identifierstyle={\ttfamily\color{black}},
% Note : highlighting of Coq identifiers is done through a new
% delimiter definition through an lstset at the begining of the
% document. Don't know how to do better.

% Style for declaration keywords
keywordstyle=[1]{\ttfamily\color{keywordcolor}},

% Style for sorts
keywordstyle=[2]{\ttfamily\color{sortcolor}},

keywordstyle=[3]{\ttfamily\color{symbolcolor}},


% Style for tactics keywords
% keywordstyle=[3]{\ttfamily\color{tacticcolor}},

% Style for attributes
% keywordstyle=[4]{\ttfamily\color{attributecolor}},

% Style for strings
stringstyle=\ttfamily,

% Style for comments
% commentstyle={\ttfamily\footnotesize },

}
